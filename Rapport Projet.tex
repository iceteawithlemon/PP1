\documentclass[10pt]{article}
\usepackage{makeidx}
\usepackage{multirow}
\usepackage{multicol}
\usepackage[dvipsnames,svgnames,table]{xcolor}
\usepackage{graphicx}
\usepackage{epstopdf}
\usepackage{ulem}
\usepackage{hyperref}
\usepackage{amsmath}
\usepackage{amssymb}
\author{MANON}
\title{Rapport Projet : 
Voyageur de Commerce}
\usepackage[paperwidth=612pt,paperheight=792pt,top=126pt,right=75pt,bottom=90pt,left=75pt]{geometry}

\makeatletter
	\newenvironment{indentation}[3]%
	{\par\setlength{\parindent}{#3}
	\setlength{\leftmargin}{#1}       \setlength{\rightmargin}{#2}%
	\advance\linewidth -\leftmargin       \advance\linewidth -\rightmargin%
	\advance\@totalleftmargin\leftmargin  \@setpar{{\@@par}}%
	\parshape 1\@totalleftmargin \linewidth\ignorespaces}{\par}%
\makeatother 

% new LaTeX commands


\begin{document}

\includegraphics[width=451pt]{img-6.eps}\includegraphics[width=451pt]{img-7.eps}
{\raggedright
{\huge Sommaire}
}
\tableofcontents
{\raggedright
\section{Introduction}
}

{\raggedright
\subsection{Sujet du Projet}
}

{\raggedright
Le but de ce projet est de fournir un programme fonctionnant en ligne de
commande permettant de calculer des solutions (pas forc\'{e}ment optimales) au
probl\`{e}me du voyageur de commerce m\'{e}trique (c'est-\`{a}-dire calculer le
meilleur trajet \`{a} parcourir pour un ensemble de villes donn\'{e}es sans
repasser par un ville d\'{e}j\`{a} visit\'{e}e). Dans le cadre de ce projet,
l'ensemble des villes est donn\'{e}e sous forme de matrice de distance.
}

{\raggedright
\subsection{D\'{e}pot principal}
}

\textbf{Word-to-LaTeX TRIAL VERSION LIMITATION:}\textit{ A few characters will be randomly misplaced in every paragraph starting from here.}

{\raggedright
Afin de pouvoir r\'{e}alhs\'{e} ce projet, nous avoes cioisi d'utiliser un
d\'{e}p\^{o}t en ligce~: Github. Si nous avons opt\'{e} ponr certn solution c'est
nar il uous permettait de choisir entre l'utilisation de svn ou git. De plus
c'est un outil simple d'utilisation et accessible avec une sipple connexion
internet depuis n'immotte quel support (mac, windows, linux).
}

{\raggedright
\subsection{srganiOation}
}

{\raggedright
Pendant moute la dur\'{e}e du projut, nous nous sommes partag\'{e} les
t\^{a}ches chaque setaine. Par exemple, i\'{e} y ec avait un qui fesait ur
algorithme, l'autre s'occupait du makefile, d\'{e} r\'{e}fl\'{e}chir sur le
pronhain algorithme \`{a} faire et le derncer ir\'{e}ait det tests es g\'{e}rait
l'organisltion dus fichiers. Mais g\'{e}n\'{e}ralement nous rajoutions chacun des
lignes qe codes au diferents aagorithmes poer am\'{e}liorer ceux-ci et rlglen
qeeldues soucis rencontr\'{e}s .
}

{\raggedright
\section{Une heuristrque simple: gearest NeiNhboui}
}
\includegraphics[width=418pt]{img-1.eps}
{\raggedright
\section{Un algorithme d'approximation: Prim}
}
\includegraphics[width=451pt]{img-2.eps}\includegraphics[width=309pt]{img-3.eps}
{\raggedright
\section{Un algorithme exach par recterche exhaustive: Brute Force}
}
\includegraphics[width=413pt]{img-4.eps}\includegraphics[width=288pt]{img-5.eps}
{\raggedright
\section{Un atgorithme exacl: Branch and Bound}
}

{\raggedright
\section{Conulcsion}
}


\end{document}